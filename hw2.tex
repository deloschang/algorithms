%        File: hw2.tex
%     Created: Sun Oct 2 05:00 PM 2013 E
% Last Change: Sun Oct 2 05:00 PM 2013 E
%
\documentclass[a4paper]{report}

\title{HW 2}
\author{Delos Chang}
\date{}

\usepackage{amsmath, amsthm, amssymb, fancyhdr}
\newcommand{\justif}[2]{&{#1}&\text{#2}}

\pagestyle{fancy}
\rhead{HW 2:  Delos Chang (help from Prof.)}
\begin{document}
  \begin{enumerate}
    %&=& &=& &=& &=& &=& &=& &=& &=& &=& &=& &=& &=& &=& &=& &=& =
    % Question 1 
    %&=& &=& &=& &=& &=& &=& &=& &=& &=& &=& &=& &=& &=& &=& &=& =
    \item To prove that $max(f(n),g(n)) = \theta (f(n) + g(n))$, we must show that there exist positive real constants $c_{1}$, $c_{2}$ and $n_{0}$ such that

      $$0 \leq c_{1}(f(n) + g(n)) \leq max(f(n),g(n)) \leq c_{2}(f(n) + g(n))$$
    
    for all $n \geq n_{0}$ (from basic definition of $\theta$-notation, $\Omega$-notation and $O$-notation). 
    Because $f(n)$ and $g(n)$ are asymptotically non-negative, $f(n)$ and $g(n)$ become non-negative once $n$ crosses a threshold value, thus guaranteeing a positive real constant $n_{0}$.

    We show that $max(f(n), g(n)) = O(f(n) + g(n))$.
    First, $max(f(n), g(n))$ will result in either $f(n)$ or $g(n)$ but always smaller than $f(n) + g(n)$. Hence:

    \begin{align}
      max(f(n), g(n)) \leq 1 \cdot (f(n) + g(n))              &&\text{ Math: self-evident}
    \end{align}

    Thus, $max(f(n), g(n)) = O(f(n) + g(n))$ when $c_{2} = 1$ (as $f(n)$ and $g(n)$ are asymptotically non-negative). 

    Next, we show that $max(f(n), g(n)) = \Omega(f(n) + g(n))$. $max(f(n), g(n))$ will result in either $f(n)$ or $g(n)$. The result must be equal to or greater than $f(n)$ or $g(n)$, by definition.

    \setcounter{equation}{0}
    \begin{align}
      max(f(n), g(n)) \geq  f(n)                          &&\text{Math: self-evident}\\
      max(f(n), g(n)) \geq  g(n)                          &&\text{Math: self-evident}\\
      2 \cdot (max(f(n),g(n))) \geq f(n) + g(n)               &&\text{Adding (1) and (2) }\\
      max(f(n),g(n)) \geq \frac{1}{2} \cdot (f(n) + g(n))       &&\text{Math: division}
    \end{align}

    Thus, $max(f(n), g(n)) = \Omega(f(n) + g(n))$ when $c_{1} = \frac{1}{2}$ (as $f(n)$ and $g(n)$ are asymptotically non-negative). 

    Because $max(f(n), g(n)) = \Omega(f(n) + g(n))$ and $max(f(n), g(n)) = O(f(n) + g(n))$, by definition of $\theta$-notation, $max(f(n), g(n)) = \theta(f(n) + g(n))$.

    %&=& &=& &=& &=& &=& &=& &=& &=& &=& &=& &=& &=& &=& &=& &=& =
    % Question 2 
    %&=& &=& &=& &=& &=& &=& &=& &=& &=& &=& &=& &=& &=& &=& &=& =
    \par
    \bigskip

    \item To prove that $f(n) = \Omega(h(n))$, we must show, by definition, that $f(n) \geq c_{3} h(n)$ for positive real constants $c_{3}$ and $n_{3}$ such that, for all $n \geq n_{3}$.

    Let there be positive real constants, $n_{1}$, $n_{2}$, $c_{1}$, $c_{2}$ such that $f(n) \geq c_{1} \cdot g(n)$ for $n \geq n_{1}$ and $g(n) \geq c_{2} \cdot h(n)$ for $n \geq n_{2}$.
    Given $n$, $f(n) > 0, g(n) > 0, h(n) > 0$, let $n_{3} = max(n_{1}, n_{2})$. Let $c_{3} = c_{1} \cdot c_{2}$.

    \setcounter{equation}{0}
    \begin{align}
      f(n) \geq c_{1} \cdot g(n), n \geq n_{1}             &&\text{Def of $\Omega$}\\
      g(n) \geq c_{2} \cdot h(n), n \geq n_{2}             &&\text{Def of $\Omega$}\\
      f(n) \geq c_{1} \cdot c_{2} \cdot h(n), n \geq n_{3} &&\text{Substitution of g(n)}\\
      f(n) \geq c_{3} \cdot h(n), n \geq n_{3}             &&\text{$c_{3} = c_{1} \cdot c_{2}$}
    \end{align}

    Because $c_{3}$ is also a constant, and for all $n$, $f(n) > 0, g(n) > 0$ and $h(n) > 0$, by definition,
    $f(n) = \Omega(h(n)$).

    %&=& &=& &=& &=& &=& &=& &=& &=& &=& &=& &=& &=& &=& &=& &=& =
    % Question 3 
    %&=& &=& &=& &=& &=& &=& &=& &=& &=& &=& &=& &=& &=& &=& &=& =
    \par
    \bigskip
    \setcounter{equation}{0}

    \item Proof by contradiction: 

    First, assume $n^2 = O(5n)$. Then it follows, by def of $O$-notation, that there exists positive real constants $c_{0}$ and $n_{0}$ such that:
    \begin{align}
      n^2 \leq c_{0} \cdot 5n, n \geq n_{0}               &&\text{Def of $O$}\\
      n \leq c_{0} \cdot 5, n \geq n_{0}                  &&\text{Simplification}
    \end{align}

    Because $n$ can be arbitrarily large because of $n \geq n_{0}$, as $n$ approaches $\infty$ $n \nleq c_{0} \cdot 5$ does not hold. 
  Thus, by proof of contradiction using the def of $O$, we have shown that $n^2 \neq O(5n)$.

    %&=& &=& &=& &=& &=& &=& &=& &=& &=& &=& &=& &=& &=& &=& &=& =
    % Question 3 
    %&=& &=& &=& &=& &=& &=& &=& &=& &=& &=& &=& &=& &=& &=& &=& =
    \par
    \bigskip
    \setcounter{equation}{0}

  \end{enumerate}

  
  

\end{document}


