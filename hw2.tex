%        File: hw2.tex
%     Created: Sun Sep 29 05:00 PM 2013 E
% Last Change: Sun Sep 29 05:00 PM 2013 E
%
\documentclass[a4paper]{report}

\title{HW 2}
\author{Delos Chang}

\usepackage{amsmath, amsthm, amssymb}
\newcommand{\justif}[2]{&{#1}&\text{#2}}

\begin{document}
  \maketitle

  \begin{enumerate}
    %&=& &=& &=& &=& &=& &=& &=& &=& &=& &=& &=& &=& &=& &=& &=& =
    % Question 1 
    %&=& &=& &=& &=& &=& &=& &=& &=& &=& &=& &=& &=& &=& &=& &=& =
    \item To prove that $max(f(n),g(n)) = \theta (f(n) + g(n))$, we must show that there exist positive real constants $c_{1}$, $c_{2}$ and $n_{0}$ such that

      $$0 \leq c_{1}(f(n) + g(n)) \leq max(f(n),g(n)) \leq c_{2}(f(n) + g(n))$$
    
    for all $n \geq n_{0}$ (from basic definition of $\theta$-notation, $\Omega$-notation and $O$-notation). 
    Because $f(n)$ and $g(n)$ are asymptotically non-negative, $f(n)$ and $g(n)$ become non-negative once $n$ crosses a threshold value, thus guaranteeing a positive real constant $n_{0}$.

    We show that $max(f(n), g(n)) = O(f(n) + g(n))$.
    First, $max(f(n), g(n))$ will result in either $f(n)$ or $g(n)$ but always smaller than $f(n) + g(n)$. Hence:

    \begin{align}
      max(f(n), g(n)) \leq 1 * (f(n) + g(n))              &&\text{ Math: self-evident}
    \end{align}

    Thus, $max(f(n), g(n)) = O(f(n) + g(n))$ when $c_{2} = 1$ (as $f(n)$ and $g(n)$ are asymptotically non-negative). 

    Next, we show that $max(f(n), g(n)) = \Omega(f(n) + g(n))$. $max(f(n), g(n))$ will result in either $f(n)$ or $g(n)$. The result must be equal to or greater than $f(n)$ or $g(n)$, by definition.

    \begin{align}
      max(f(n), g(n)) \geq  f(n)                          &&\text{Math: self-evident}\\
      max(f(n), g(n)) \geq  g(n)                          &&\text{Math: self-evident}\\
      2 * (max(f(n),g(n))) \geq f(n) + g(n)               &&\text{Adding (2) and (3) }\\
      max(f(n),g(n)) \geq \frac{1}{2}*(f(n) + g(n))       &&\text{Math: division}
    \end{align}

    Thus, $max(f(n), g(n)) = \Omega(f(n) + g(n))$ when $c_{1} = \frac{1}{2}$ (as $f(n)$ and $g(n)$ are asymptotically non-negative). 

    Because $max(f(n), g(n)) = \Omega(f(n) + g(n))$ and $max(f(n), g(n)) = O(f(n) + g(n))$, by definition of $\theta$-notation, $max(f(n), g(n)) = \theta(f(n) + g(n))$.

    %&=& &=& &=& &=& &=& &=& &=& &=& &=& &=& &=& &=& &=& &=& &=& =
    % Question 2 
    %&=& &=& &=& &=& &=& &=& &=& &=& &=& &=& &=& &=& &=& &=& &=& =
    \par
    \bigskip

    \item 

  \end{enumerate}

  
  

\end{document}


