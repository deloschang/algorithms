%        File: hw2.tex
%     Created: Sun Oct 2 05:00 PM 2013 E
% Last Change: Sun Oct 2 05:00 PM 2013 E
%
\documentclass[a4paper]{report}

\title{HW 2}
\author{Delos Chang}
\date{}

\usepackage{amsmath, amsthm, amssymb, fancyhdr}
\newcommand{\justif}[2]{&{#1}&\text{#2}}

\pagestyle{fancy}
\rhead{HW 2:  Delos Chang (help from Prof.)}
\begin{document}
  \begin{enumerate}
    %&=& &=& &=& &=& &=& &=& &=& &=& &=& &=& &=& &=& &=& &=& &=& =
    % Question 1 
    %&=& &=& &=& &=& &=& &=& &=& &=& &=& &=& &=& &=& &=& &=& &=& =
    \item To prove that $max(f(n),g(n)) = \theta (f(n) + g(n))$, we must show that there exist positive real constants $c_{1}$, $c_{2}$ and $n_{0}$ such that

      $$0 \leq c_{1}(f(n) + g(n)) \leq max(f(n),g(n)) \leq c_{2}(f(n) + g(n))$$
    
    for all $n \geq n_{0}$ (from basic definition of $\theta$-notation, $\Omega$-notation and $O$-notation). 
    Because $f(n)$ and $g(n)$ are asymptotically non-negative, $f(n)$ and $g(n)$ become non-negative once $n$ crosses a threshold value, thus guaranteeing a positive real constant $n_{0}$.

    We show that $max(f(n), g(n)) = O(f(n) + g(n))$.
    First, $max(f(n), g(n))$ will result in either $f(n)$ or $g(n)$ but always smaller than $f(n) + g(n)$. Hence:

    \begin{align}
      max(f(n), g(n)) \leq 1 \cdot (f(n) + g(n))              &&\text{ Math: self-evident}
    \end{align}

    Thus, $max(f(n), g(n)) = O(f(n) + g(n))$ when $c_{2} = 1$ (as $f(n)$ and $g(n)$ are asymptotically non-negative). 

    Next, we show that $max(f(n), g(n)) = \Omega(f(n) + g(n))$. $max(f(n), g(n))$ will result in either $f(n)$ or $g(n)$. The result must be equal to or greater than $f(n)$ or $g(n)$, by definition.

    \setcounter{equation}{0}
    \begin{align}
      max(f(n), g(n)) \geq  f(n)                          &&\text{Math: self-evident}\\
      max(f(n), g(n)) \geq  g(n)                          &&\text{Math: self-evident}\\
      2 \cdot (max(f(n),g(n))) \geq f(n) + g(n)               &&\text{Adding (1) and (2) }\\
      max(f(n),g(n)) \geq \frac{1}{2} \cdot (f(n) + g(n))       &&\text{Math: division}
    \end{align}

    Thus, $max(f(n), g(n)) = \Omega(f(n) + g(n))$ when $c_{1} = \frac{1}{2}$ (as $f(n)$ and $g(n)$ are asymptotically non-negative). 

    Because $max(f(n), g(n)) = \Omega(f(n) + g(n))$ and $max(f(n), g(n)) = O(f(n) + g(n))$, by definition of $\theta$-notation, $max(f(n), g(n)) = \theta(f(n) + g(n))$.

    %&=& &=& &=& &=& &=& &=& &=& &=& &=& &=& &=& &=& &=& &=& &=& =
    % Question 2 
    %&=& &=& &=& &=& &=& &=& &=& &=& &=& &=& &=& &=& &=& &=& &=& =
    \par
    \bigskip

    \item To prove that $f(n) = \Omega(h(n))$, we must show, by definition, that $f(n) \geq c_{3} h(n)$ for positive real constants $c_{3}$ and $n_{3}$ such that, for all $n \geq n_{3}$.

    Let there be positive real constants, $n_{1}$, $n_{2}$, $c_{1}$, $c_{2}$ such that $f(n) \geq c_{1} \cdot g(n)$ for $n \geq n_{1}$ and $g(n) \geq c_{2} \cdot h(n)$ for $n \geq n_{2}$.
    Given $n$, $f(n) > 0, g(n) > 0, h(n) > 0$, let $n_{3} = max(n_{1}, n_{2})$. Let $c_{3} = c_{1} \cdot c_{2}$.

    \setcounter{equation}{0}
    \begin{align}
      f(n) \geq c_{1} \cdot g(n), n \geq n_{1}             &&\text{Def of $\Omega$}\\
      g(n) \geq c_{2} \cdot h(n), n \geq n_{2}             &&\text{Def of $\Omega$}\\
      f(n) \geq c_{1} \cdot c_{2} \cdot h(n), n \geq n_{3} &&\text{Substitution of g(n)}\\
      f(n) \geq c_{3} \cdot h(n), n \geq n_{3}             &&\text{$c_{3} = c_{1} \cdot c_{2}$}
    \end{align}

    Because $c_{3}$ is also a constant, and for all $n$, $f(n) > 0, g(n) > 0$ and $h(n) > 0$, by definition,
    $f(n) = \Omega(h(n)$).

    %&=& &=& &=& &=& &=& &=& &=& &=& &=& &=& &=& &=& &=& &=& &=& =
    % Question 3 
    %&=& &=& &=& &=& &=& &=& &=& &=& &=& &=& &=& &=& &=& &=& &=& =
    \par
    \bigskip
    \setcounter{equation}{0}

    \item Proof by contradiction: 

    First, assume $n^2 = O(5n)$. Then it follows, by def of $O$-notation, that there exists positive real constants $c_{0}$ and $n_{0}$ such that:
    \begin{align}
      n^2 \leq c_{0} \cdot 5n, n \geq n_{0}               &&\text{Def of $O$}\\
      n \leq c_{0} \cdot 5, n \geq n_{0}                  &&\text{Simplification}
    \end{align}

    Because $n$ can be arbitrarily large because of $n \geq n_{0}$, as $n$ approaches $\infty$ $n \nleq c_{0} \cdot 5$ does not hold. 
  Thus, by proof of contradiction using the def of $O$, we have shown that $n^2 \neq O(5n)$.

    %&=& &=& &=& &=& &=& &=& &=& &=& &=& &=& &=& &=& &=& &=& &=& =
    % Question 4 
    %&=& &=& &=& &=& &=& &=& &=& &=& &=& &=& &=& &=& &=& &=& &=& =
    \par
    \bigskip
    \setcounter{equation}{0}
    
    \item Let there be positive real constants $c_{0}$ and $n_{0}$, $n \geq n_{0}$. Also, let $c_{1} = \frac{1}{c_{0}}$. By definition of $O$-notation, 

    Assuming $f(n)$ and $g(n)$ are asymptotically non-negative: 
    \begin{align}
      f'(n) = O(f(n))                                   &&\text{Given }\\
      f'(n) \leq c_{0} \cdot f(n), n \geq n_{0}          &&\text{Def of $O$}\\
      \frac{1}{c_{0}} \cdot f'(n) \leq \cdot f(n), n \geq n_{0}          &&\text{Math}\\
      f(n) \geq c_{1} \cdot f'(n), n \geq n_{0}          &&\text{Math: $c_{1} = \frac{1}{c_{0}}$}\\
      f(n)  = \Omega(f'(n))                               &&\text{Def of $\Omega$}
    \end{align}

    
    %&=& &=& &=& &=& &=& &=& &=& &=& &=& &=& &=& &=& &=& &=& &=& =
    % Question 5 
    %&=& &=& &=& &=& &=& &=& &=& &=& &=& &=& &=& &=& &=& &=& &=& =
    \pagebreak
    \bigskip
    \setcounter{equation}{0}

    \item 
      % a)

      Let all following log bases not be 1. 

      a) Fact: If log $f(n) = o($log $g(n))$, then $f(n) = o(g(n))$. 
      Using this fact, let $f(n) = 2^n$ and $g(n) = n!$. Thus, taking logs of both sides:

      \begin{align}
        log(2^n) = n \cdot log 2 \\
        log(2^n) = n \\
        log(n!) = \theta(n \cdot log n) 
      \end{align}

      Because $\theta(n \cdot$ log $n) = \omega(n)$, it follows that $2^n = o(n!)$.

      \bigskip
      % b)
      b) Fact: $a^{log_{b} c}$ = $c ^ {log_{b} a} $. Thus:
      \begin{align}
        6^{log n} = n^{log 6} \\
        5^{log n} = n^{log 5} 
      \end{align}

      Because $log 6 > log 5$, $n^{log 5}$ = $o(n^{log 6})$.

      \bigskip
      % c)
      c) The co-domain of a sine function is [-1, 1]. Thus, for all $n$, $n^{-1} \leq n^{sin n} \leq n$.

      Thus, depending on $n$, sometimes $n^{sin n} > \sqrt{n}$, sometimes $n^{sin n} < \sqrt{n}$.
      It follows that there are no positive real constants $n_{0}, c$ such that for all $n \geq n_{0}$,  $n^{sin n} \geq c \cdot \sqrt{n}$.
      For every $n_{1}$ where $n_{1}^{sin n_{1}} \geq c \cdot \sqrt{n_{1}}$, there exist a $n_{2}$ where $n_{2} > n_{1}$ and $n_{2}^{sin n_{2}} \leq c \cdot \sqrt{n_{2}}$.

      For the same reason, there are no positive real constants $n_{0}$, $c$ such that for all $n \geq n_{0}$, 
      $n^{sin n} \leq c \cdot \sqrt{n}$.

      Hence, there is no asymptotic relationship to be described.

      \bigskip
      % d)
      d) 
      Consider when $n$ is even and $n > 6$, $f(n) = 6n$ so $n^2 > f(n)$. Thus, there are an infinite number of $n$'s where $n^2 > f(n)$.
      Similarly, when $n$ is odd and $n > 0$, $f(n) = 5n^2$ so $n^2 < f(n)$. Thus, there are an infinite number of $n$'s where $n^2 < f(n)$. 

      Because there are infinite numbers of $n$ where $f(n)$ can be both $<$ and $>$ than $n^2$, there are no positive real constants $n_{0}$ and $c$ such that 
      for all $n \geq n_{0}$, $f(n) \geq c \cdot n^2$.

      For the same reason, there are no positive real constants $n_{0}$, $c$ such that for all $n \geq n_{0}$, 
      $f(n) \leq c \cdot n^2$.

      Hence, there is no asymptotic relationship to be described.

      \bigskip
      % e)
      e) Fact: If log $f(n) = o($log $g(n))$, then $f(n) = o(g(n))$. 
      Using this fact, let $f(n) = (n!)^{n}$ and $g(n) = 2^(2n)$. Thus, taking logs of both sides:

      \begin{align}
        log(n!^{n}) = n \cdot log(n!)\\
        log(n!^{n}) = n \cdot \theta(n \cdot log n)\\
        log(n!^{n}) = \theta(n^2 \cdot log n)\\
        log(2^{2n}) = 2n 
      \end{align}

      Since $\theta(n^2 log n) = \omega(2n)$, it follows from the aforementioned fact that:

      $2^{2n} = o((n!)^{n})$

    %&=& &=& &=& &=& &=& &=& &=& &=& &=& &=& &=& &=& &=& &=& &=& =
    % Question 6 
    %&=& &=& &=& &=& &=& &=& &=& &=& &=& &=& &=& &=& &=& &=& &=& =
    \pagebreak
    \bigskip
    \setcounter{equation}{0}
    \item We disprove the statement by considering a counterexample where log $f(n) = O($log $g(n))$ but $f(n) \neq O(g(n))$: 
    
      Consider $f(n) = n^3$ and $g(n) = n$. 

      \begin{align}
        log(f(n)) = 3 \cdot log(n)                    &&\text{Taking log of f(n)}\\
        log(g(n)) = log(n)                            &&\text{Taking log of g(n)}\\
        log(f(n)) = O(log(n))                         &&\text{Def of $O$}
      \end{align}

      We have shown first part of the implication. But it is evident that $n = O(n^3)$.
      Thus, while log $f(n) = O($log $g(n))$, $f(n) \neq O(g(n))$, disproving the statement.

    %&=& &=& &=& &=& &=& &=& &=& &=& &=& &=& &=& &=& &=& &=& &=& =
    % Question 7 
    %&=& &=& &=& &=& &=& &=& &=& &=& &=& &=& &=& &=& &=& &=& &=& =
    \bigskip
    \setcounter{equation}{0}
    \item We disprove the statement by considering a counterexample: 

      Consider $f(n)$ where $f$ is defined as follows: $f(n) = 1$ if $n$ is even; else $f(n) = n^{-1}$.
      Let $c = 1$ and $n_{0} = 1$. Then:

      $f(n) \leq c \cdot g(n)$ for all $n \geq n_{0}$.
      Simplifying $f(n) \leq 1$ for all $n \geq n_{0}$, which is true because the biggest value $f(n)$ can be is 1.

      Thus, by definition, $f(n) = O(g(n))$.

      Next, we show that $f(n) \neq \Omega(g(n))$.

      Because $f(n) = n^{-1}$ when $n$ is odd, there are infinite n's that dip below any constant infinitely often.
      In other words, as n approaches infinity, $f(n)$ oscillates between 1 and falls below any constant infinitely. 
      Thus, no positive real constant $n_{0}$ exists such that $f(n) \geq c \cdot g(n)$ for all $n \geq n_{0}$.

      Hence, by definition, $f(n) \neq \Omega(g(n))$. 
      
      Finally, we show that $f(n) \neq o(g(n))$.
      For $f(n) = o(g(n))$ to be true, for {\bf any} positive constant $c > 0$, there must exist a constant $n_{0} > 0$ such that
      $0 \leq f(n) < c \cdot g(n)$ for all $n \geq n_{0}$. 

      Consider $c = 0.4$. $g(n) = 1$ everywhere and $f(n) > 0.4 * 1$ infinitely many times as n approaches infinity 
      (when n is even). Thus, there is no $n_{0} > 0 $ such that $0 \leq f(n) < c \cdot g(n)$ for all $n \geq n_{0}$.  
      
      Because we have shown this condition to fail for $c = 0.4$, $f(n) \neq o(g(n))$.

      Hence, we have disproved the statement.

  \end{enumerate}

  
  

\end{document}


